\documentclass[12pt,a4paper]{article}
\usepackage[utf8]{inputenc}
\usepackage[vietnamese]{babel}
\usepackage{amsmath}
\usepackage{graphicx}
\usepackage{hyperref}
\usepackage{geometry}

\geometry{
  a4paper,
  total={170mm,257mm},
  left=20mm,
  top=20mm,
}

\title{Báo cáo Bài tập lớn: Dự đoán Giá xe Audi}
\author{Nhóm [Tên Nhóm]}
\date{\today}

\begin{document}

\maketitle

\begin{abstract}
Tóm tắt nội dung báo cáo...
\end{abstract}

\tableofcontents
\newpage

\section{Introduction (Giới thiệu)}
% TODO: [Thành_viên_C] write this section
% - Giới thiệu bài toán dự đoán giá xe cũ.
% - Tầm quan trọng của bài toán.
% - Giới thiệu dataset Audi.

\section{Data Analysis (Phân tích dữ liệu)}
% TODO: [Thành_viên_A & B] write this section
% - Mô tả các trường dữ liệu (features).
% - Trực quan hóa dữ liệu (EDA): Phân bố giá, tương quan giữa các biến.
% - Tiền xử lý dữ liệu (Preprocessing): Encoding, Scaling, Missing values.

\section{Methodology (Phương pháp)}

\subsection{Linear Regression (Hồi quy tuyến tính)}
% Baseline model implemented by Leader.
% Mô tả ngắn gọn về Linear Regression.

\subsection{Support Vector Machine (SVM)}
% TODO: [Thành_viên_A] write this section
% - Lý thuyết cơ bản về SVR.
% - Các tham số đã sử dụng (Kernel, C, Epsilon).

\subsection{Random Forest}
% TODO: [Thành_viên_B] write this section
% - Lý thuyết về Random Forest.
% - Quá trình Hyperparameter Tuning (GridSearchCV).

\subsection{Deep Learning (MLPRegressor)}
% TODO: [Thắng (Leader)] write this section
% - Kiến trúc mạng Neural Network (MLP).
% - Cấu hình các lớp ẩn (Hidden layers), hàm kích hoạt (Activation function).

\subsection{Ensemble Learning (Voting Regressor)}
% TODO: [Thắng (Leader)] write this section
% - Phương pháp kết hợp các mô hình (Voting).

\section{Experimental Results (Kết quả thực nghiệm)}
% TODO: [Thắng (Leader)] write this section
% - Bảng so sánh kết quả (R2 Score, RMSE, MAE) của các mô hình.
% - Biểu đồ so sánh hiệu năng.
% - Nhận xét và đánh giá mô hình tốt nhất.

\section{Conclusion (Kết luận)}
% Tổng kết lại kết quả đạt được và hướng phát triển.

\begin{thebibliography}{9}
% TODO: [Thành_viên_C] Add references here
\bibitem{sklearn} Scikit-learn documentation.
\end{thebibliography}

\end{document}
